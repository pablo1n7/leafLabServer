% Generated by Sphinx.
\def\sphinxdocclass{report}
\documentclass[a4paper,10pt,oneside]{sphinxmanual}
\usepackage[utf8]{inputenc}
\DeclareUnicodeCharacter{00A0}{\nobreakspace}
\usepackage{cmap}
\usepackage[T1]{fontenc}
\usepackage[spanish]{babel}
\usepackage{times}
\usepackage[Sonny]{fncychap}
\usepackage{longtable}
\usepackage{sphinx}
\usepackage{multirow}

\addto\captionsspanish{\renewcommand{\figurename}{Figura }}
\addto\captionsspanish{\renewcommand{\tablename}{Tabla }}
\floatname{literal-block}{Lista }



\title{Aplicaciones móviles multiplaforma sensibles al contexto: Una aplicación científica para el relevamiento flóristico.}
\date{25 de August de 2015}
\release{1.0}
\author{}
\newcommand{\sphinxlogo}{}
\renewcommand{\releasename}{Publicación}
\makeindex

\makeatletter
\def\PYG@reset{\let\PYG@it=\relax \let\PYG@bf=\relax%
    \let\PYG@ul=\relax \let\PYG@tc=\relax%
    \let\PYG@bc=\relax \let\PYG@ff=\relax}
\def\PYG@tok#1{\csname PYG@tok@#1\endcsname}
\def\PYG@toks#1+{\ifx\relax#1\empty\else%
    \PYG@tok{#1}\expandafter\PYG@toks\fi}
\def\PYG@do#1{\PYG@bc{\PYG@tc{\PYG@ul{%
    \PYG@it{\PYG@bf{\PYG@ff{#1}}}}}}}
\def\PYG#1#2{\PYG@reset\PYG@toks#1+\relax+\PYG@do{#2}}

\expandafter\def\csname PYG@tok@gd\endcsname{\def\PYG@tc##1{\textcolor[rgb]{0.63,0.00,0.00}{##1}}}
\expandafter\def\csname PYG@tok@gu\endcsname{\let\PYG@bf=\textbf\def\PYG@tc##1{\textcolor[rgb]{0.50,0.00,0.50}{##1}}}
\expandafter\def\csname PYG@tok@gt\endcsname{\def\PYG@tc##1{\textcolor[rgb]{0.00,0.27,0.87}{##1}}}
\expandafter\def\csname PYG@tok@gs\endcsname{\let\PYG@bf=\textbf}
\expandafter\def\csname PYG@tok@gr\endcsname{\def\PYG@tc##1{\textcolor[rgb]{1.00,0.00,0.00}{##1}}}
\expandafter\def\csname PYG@tok@cm\endcsname{\let\PYG@it=\textit\def\PYG@tc##1{\textcolor[rgb]{0.25,0.50,0.56}{##1}}}
\expandafter\def\csname PYG@tok@vg\endcsname{\def\PYG@tc##1{\textcolor[rgb]{0.73,0.38,0.84}{##1}}}
\expandafter\def\csname PYG@tok@m\endcsname{\def\PYG@tc##1{\textcolor[rgb]{0.13,0.50,0.31}{##1}}}
\expandafter\def\csname PYG@tok@mh\endcsname{\def\PYG@tc##1{\textcolor[rgb]{0.13,0.50,0.31}{##1}}}
\expandafter\def\csname PYG@tok@cs\endcsname{\def\PYG@tc##1{\textcolor[rgb]{0.25,0.50,0.56}{##1}}\def\PYG@bc##1{\setlength{\fboxsep}{0pt}\colorbox[rgb]{1.00,0.94,0.94}{\strut ##1}}}
\expandafter\def\csname PYG@tok@ge\endcsname{\let\PYG@it=\textit}
\expandafter\def\csname PYG@tok@vc\endcsname{\def\PYG@tc##1{\textcolor[rgb]{0.73,0.38,0.84}{##1}}}
\expandafter\def\csname PYG@tok@il\endcsname{\def\PYG@tc##1{\textcolor[rgb]{0.13,0.50,0.31}{##1}}}
\expandafter\def\csname PYG@tok@go\endcsname{\def\PYG@tc##1{\textcolor[rgb]{0.20,0.20,0.20}{##1}}}
\expandafter\def\csname PYG@tok@cp\endcsname{\def\PYG@tc##1{\textcolor[rgb]{0.00,0.44,0.13}{##1}}}
\expandafter\def\csname PYG@tok@gi\endcsname{\def\PYG@tc##1{\textcolor[rgb]{0.00,0.63,0.00}{##1}}}
\expandafter\def\csname PYG@tok@gh\endcsname{\let\PYG@bf=\textbf\def\PYG@tc##1{\textcolor[rgb]{0.00,0.00,0.50}{##1}}}
\expandafter\def\csname PYG@tok@ni\endcsname{\let\PYG@bf=\textbf\def\PYG@tc##1{\textcolor[rgb]{0.84,0.33,0.22}{##1}}}
\expandafter\def\csname PYG@tok@nl\endcsname{\let\PYG@bf=\textbf\def\PYG@tc##1{\textcolor[rgb]{0.00,0.13,0.44}{##1}}}
\expandafter\def\csname PYG@tok@nn\endcsname{\let\PYG@bf=\textbf\def\PYG@tc##1{\textcolor[rgb]{0.05,0.52,0.71}{##1}}}
\expandafter\def\csname PYG@tok@no\endcsname{\def\PYG@tc##1{\textcolor[rgb]{0.38,0.68,0.84}{##1}}}
\expandafter\def\csname PYG@tok@na\endcsname{\def\PYG@tc##1{\textcolor[rgb]{0.25,0.44,0.63}{##1}}}
\expandafter\def\csname PYG@tok@nb\endcsname{\def\PYG@tc##1{\textcolor[rgb]{0.00,0.44,0.13}{##1}}}
\expandafter\def\csname PYG@tok@nc\endcsname{\let\PYG@bf=\textbf\def\PYG@tc##1{\textcolor[rgb]{0.05,0.52,0.71}{##1}}}
\expandafter\def\csname PYG@tok@nd\endcsname{\let\PYG@bf=\textbf\def\PYG@tc##1{\textcolor[rgb]{0.33,0.33,0.33}{##1}}}
\expandafter\def\csname PYG@tok@ne\endcsname{\def\PYG@tc##1{\textcolor[rgb]{0.00,0.44,0.13}{##1}}}
\expandafter\def\csname PYG@tok@nf\endcsname{\def\PYG@tc##1{\textcolor[rgb]{0.02,0.16,0.49}{##1}}}
\expandafter\def\csname PYG@tok@si\endcsname{\let\PYG@it=\textit\def\PYG@tc##1{\textcolor[rgb]{0.44,0.63,0.82}{##1}}}
\expandafter\def\csname PYG@tok@s2\endcsname{\def\PYG@tc##1{\textcolor[rgb]{0.25,0.44,0.63}{##1}}}
\expandafter\def\csname PYG@tok@vi\endcsname{\def\PYG@tc##1{\textcolor[rgb]{0.73,0.38,0.84}{##1}}}
\expandafter\def\csname PYG@tok@nt\endcsname{\let\PYG@bf=\textbf\def\PYG@tc##1{\textcolor[rgb]{0.02,0.16,0.45}{##1}}}
\expandafter\def\csname PYG@tok@nv\endcsname{\def\PYG@tc##1{\textcolor[rgb]{0.73,0.38,0.84}{##1}}}
\expandafter\def\csname PYG@tok@s1\endcsname{\def\PYG@tc##1{\textcolor[rgb]{0.25,0.44,0.63}{##1}}}
\expandafter\def\csname PYG@tok@gp\endcsname{\let\PYG@bf=\textbf\def\PYG@tc##1{\textcolor[rgb]{0.78,0.36,0.04}{##1}}}
\expandafter\def\csname PYG@tok@sh\endcsname{\def\PYG@tc##1{\textcolor[rgb]{0.25,0.44,0.63}{##1}}}
\expandafter\def\csname PYG@tok@ow\endcsname{\let\PYG@bf=\textbf\def\PYG@tc##1{\textcolor[rgb]{0.00,0.44,0.13}{##1}}}
\expandafter\def\csname PYG@tok@sx\endcsname{\def\PYG@tc##1{\textcolor[rgb]{0.78,0.36,0.04}{##1}}}
\expandafter\def\csname PYG@tok@bp\endcsname{\def\PYG@tc##1{\textcolor[rgb]{0.00,0.44,0.13}{##1}}}
\expandafter\def\csname PYG@tok@c1\endcsname{\let\PYG@it=\textit\def\PYG@tc##1{\textcolor[rgb]{0.25,0.50,0.56}{##1}}}
\expandafter\def\csname PYG@tok@kc\endcsname{\let\PYG@bf=\textbf\def\PYG@tc##1{\textcolor[rgb]{0.00,0.44,0.13}{##1}}}
\expandafter\def\csname PYG@tok@c\endcsname{\let\PYG@it=\textit\def\PYG@tc##1{\textcolor[rgb]{0.25,0.50,0.56}{##1}}}
\expandafter\def\csname PYG@tok@mf\endcsname{\def\PYG@tc##1{\textcolor[rgb]{0.13,0.50,0.31}{##1}}}
\expandafter\def\csname PYG@tok@err\endcsname{\def\PYG@bc##1{\setlength{\fboxsep}{0pt}\fcolorbox[rgb]{1.00,0.00,0.00}{1,1,1}{\strut ##1}}}
\expandafter\def\csname PYG@tok@mb\endcsname{\def\PYG@tc##1{\textcolor[rgb]{0.13,0.50,0.31}{##1}}}
\expandafter\def\csname PYG@tok@ss\endcsname{\def\PYG@tc##1{\textcolor[rgb]{0.32,0.47,0.09}{##1}}}
\expandafter\def\csname PYG@tok@sr\endcsname{\def\PYG@tc##1{\textcolor[rgb]{0.14,0.33,0.53}{##1}}}
\expandafter\def\csname PYG@tok@mo\endcsname{\def\PYG@tc##1{\textcolor[rgb]{0.13,0.50,0.31}{##1}}}
\expandafter\def\csname PYG@tok@kd\endcsname{\let\PYG@bf=\textbf\def\PYG@tc##1{\textcolor[rgb]{0.00,0.44,0.13}{##1}}}
\expandafter\def\csname PYG@tok@mi\endcsname{\def\PYG@tc##1{\textcolor[rgb]{0.13,0.50,0.31}{##1}}}
\expandafter\def\csname PYG@tok@kn\endcsname{\let\PYG@bf=\textbf\def\PYG@tc##1{\textcolor[rgb]{0.00,0.44,0.13}{##1}}}
\expandafter\def\csname PYG@tok@o\endcsname{\def\PYG@tc##1{\textcolor[rgb]{0.40,0.40,0.40}{##1}}}
\expandafter\def\csname PYG@tok@kr\endcsname{\let\PYG@bf=\textbf\def\PYG@tc##1{\textcolor[rgb]{0.00,0.44,0.13}{##1}}}
\expandafter\def\csname PYG@tok@s\endcsname{\def\PYG@tc##1{\textcolor[rgb]{0.25,0.44,0.63}{##1}}}
\expandafter\def\csname PYG@tok@kp\endcsname{\def\PYG@tc##1{\textcolor[rgb]{0.00,0.44,0.13}{##1}}}
\expandafter\def\csname PYG@tok@w\endcsname{\def\PYG@tc##1{\textcolor[rgb]{0.73,0.73,0.73}{##1}}}
\expandafter\def\csname PYG@tok@kt\endcsname{\def\PYG@tc##1{\textcolor[rgb]{0.56,0.13,0.00}{##1}}}
\expandafter\def\csname PYG@tok@sc\endcsname{\def\PYG@tc##1{\textcolor[rgb]{0.25,0.44,0.63}{##1}}}
\expandafter\def\csname PYG@tok@sb\endcsname{\def\PYG@tc##1{\textcolor[rgb]{0.25,0.44,0.63}{##1}}}
\expandafter\def\csname PYG@tok@k\endcsname{\let\PYG@bf=\textbf\def\PYG@tc##1{\textcolor[rgb]{0.00,0.44,0.13}{##1}}}
\expandafter\def\csname PYG@tok@se\endcsname{\let\PYG@bf=\textbf\def\PYG@tc##1{\textcolor[rgb]{0.25,0.44,0.63}{##1}}}
\expandafter\def\csname PYG@tok@sd\endcsname{\let\PYG@it=\textit\def\PYG@tc##1{\textcolor[rgb]{0.25,0.44,0.63}{##1}}}

\def\PYGZbs{\char`\\}
\def\PYGZus{\char`\_}
\def\PYGZob{\char`\{}
\def\PYGZcb{\char`\}}
\def\PYGZca{\char`\^}
\def\PYGZam{\char`\&}
\def\PYGZlt{\char`\<}
\def\PYGZgt{\char`\>}
\def\PYGZsh{\char`\#}
\def\PYGZpc{\char`\%}
\def\PYGZdl{\char`\$}
\def\PYGZhy{\char`\-}
\def\PYGZsq{\char`\'}
\def\PYGZdq{\char`\"}
\def\PYGZti{\char`\~}
% for compatibility with earlier versions
\def\PYGZat{@}
\def\PYGZlb{[}
\def\PYGZrb{]}
\makeatother

\renewcommand\PYGZsq{\textquotesingle}

\begin{document}
\shorthandoff{"}
\maketitle
\tableofcontents
\phantomsection\label{index::doc}



\chapter{Preambulo}
\label{index:preambulo}\label{index:welcome-to-tesina-leaflab-s-documentation}

\section{This is a Title}
\label{resumen::doc}\label{resumen:this-is-a-title}
That has a paragraph about a main subject and is set when the `='
is at least the same length of the title itself.


\subsection{Subject Subtitle}
\label{resumen:subject-subtitle}
Subtitles are set with `-` and are required to have the same length
of the subtitle itself, just like titles.

Lists can be unnumbered like:
\begin{itemize}
\item {} 
Item Foo

\item {} 
Item Bar

\end{itemize}

Or automatically numbered:
\begin{enumerate}
\item {} 
Item 1

\item {} 
Item 2

\end{enumerate}


\subsection{Inline Markup}
\label{resumen:inline-markup}
Words can have \emph{emphasis in italics} or be \textbf{bold} and you can define
code samples with back quotes, like when you talk about a command: \code{sudo}
gives you super user powers!


\section{Abstract}
\label{abstract:abstract}\label{abstract::doc}
This abstract.


\section{Motivación}
\label{motivacion:motivacion}\label{motivacion::doc}
Durante nuestros años de estudio en la carrera, una de las mayores inquietudes del grupo fue la de lograr un producto cuyos límites no solo se encuentren en el área curricular. Como es sabido la informática nació para dar apoyo a las diferentes áreas de la ciencia, y realizar un aporte a otra disciplina era una cuota pendiente para el grupo.

Es por esta razón que el presente trabajo de tesina no solo se centra en el desarrollo de una aplicación para el ámbito académico, sino ser una herramienta operacional que, además de concluir una etapa de formación pueda servir como apoyo a otra rama de la ciencia. Con esto en mente, y tomando conocimiento de la necesidad de los biólogos de la zona, se comenzó el desarrollo en conjunto de una aplicación móvil multiplataforma sensible al contexto destinada a tal fin.


\section{Introducción}
\label{introduccion:introduccion}\label{introduccion::doc}
El desarrollo propuesto se basa en la elaboración de una aplicación multiplataforma sensible al contexto  que haciendo uso de las tecnologıas actuales permita a los cientıficos de las ciencias biológicas especializados en Botánica, realizar las tareas de relevamiento florístico correspondientes a su actividad científica, con un soporte informático adecuado. El desarrollo de la aplicación se llevará a cabo haciendo uso de las tecnologıas móviles que el grupo considera adecuadas para generar un software sensible a su actividad y contexto de trabajo particular, pudiendo ser utilizado en una gran variedad de dispositivos. Este trabajo también contempla la construcción de un software adecuado para la sincronización y reunión de los datos recolectados en las campaña de relevamiento en un único servidor de bases de datos que los científicos podrán acceder y navegar a través de una aplicación web que sería desarrollada para tal fin.

En este punto es necesario introducir conceptos básicos como lo es el de Aplicaciones Multiplataforma. Se trata de un tipo de aplicación que puede ser ejecutada en diferentes sistemas. De estas aplicaciones se derivan dos grandes ramas: aquellas que deben ser compiladas individualmente para cada plataforma en que se quiera ejecutar, y aquellas que pueden ser ejecutadas directamente en cualquier plataforma sin ninguna preparación especial.

Otro concepto importante a introducir es el de sensibilidad al contexto. Las aplicaciones sensibles al contexto son aquellas que adaptan automáticamente su comportamiento y configuración, dependiendo de las condiciones del entorno y de las preferencias del usuario, sin que el mismo deba mediar, o lo haga lo menos posible, con la aplicación. En la actualidad, los dispositivos móviles son capaces de sensar propiedades del contexto que pueden ser de mucha ayuda si se las utiliza para “intuir” las intenciones de los usuarios con el dispositivo.

Este tipo de aplicaciones intenta liberar al usuario de la necesidad de interactuar continuamente con el dispositivo para indicar que debe hacer. La idea es que un usuario pueda desempeñar sus actividades normales, y su dispositivo “responda” junto con él a las situaciones que se presentan.

Todos los conceptos anteriormente mencionados y más serán abordados en profundidad en capítulos posteriores.


\section{Objetivos}
\label{objetivos::doc}\label{objetivos:objetivos}
Al comenzar este trabajo, fue necesario plantear los objetivos que se quiere alcanzar con el desarrollo del mismo, teniendo siempre en vista la problemática a resolver. Se categorizó los objetivos propuestos de la siguiente manera:


\subsection{Objetivos Generales}
\label{objetivos:objetivos-generales}
Los objetivos de la presente tesina contempla el estudio de la computación móvil sensible al contexto y las tecnologıas existentes para el desarrollo de productos de software móvil multiplataforma con el fin de construir una aplicación de relevancia para el ámbito cientıfico de referencia.

También se pretende que el producto resultante pueda ser utilizado como una herramienta habitual a la hora de hacer el trabajo de campo en el área de botánica para el que está dirigido.

Un último objetivo general, es el de lograr un producto en el que se reúna el conocimiento adquirido durante los años de estudio de la carrera de manera coherente y con un fin no meramente profesional.


\subsection{Objetivos Particulares}
\label{objetivos:objetivos-particulares}\begin{itemize}
\item {} 
Estudiar y documentar el marco teórico de referencia.

\item {} 
Analizar el estado actual del mercado en relación a las herramientas. de producción de software móvil multiplataforma sensible al contexto.

\item {} 
Construir una aplicación móvil multiplataforma y sensible al contexto de utilización cientıfica para el relevamiento florıstico.

\item {} 
Construir un software adecuado para la sincronización y reunión de los datos recolectados en las campaña de relevamiento.

\item {} 
Investigar distintos lenguajes y ambientes de desarrollo, para determinar los que mejor se adapten al sistema que se quiere lograr.

\item {} 
Evaluar el estado de madurez de las herramientas investigadas para el desarrollo de aplicaciones con estas características.

\end{itemize}


\section{Límites}
\label{limites::doc}\label{limites:limites}\begin{itemize}
\item {} 
El proyecto se limita a la recolección, digitalización y visualización simple de los datos. El procesamiento de los mismos, se encuentra aún en dia en constante evolución, por lo cual no es aceptable fijar los datos a una manera específica de procesamiento.

\end{itemize}


\chapter{Estado de Arte}
\label{index:estado-de-arte}
asdasdasdasasdasdasddsa


\section{This is a Title}
\label{resumen::doc}\label{resumen:this-is-a-title}
That has a paragraph about a main subject and is set when the `='
is at least the same length of the title itself.


\subsection{Subject Subtitle}
\label{resumen:subject-subtitle}
Subtitles are set with `-` and are required to have the same length
of the subtitle itself, just like titles.

Lists can be unnumbered like:
\begin{itemize}
\item {} 
Item Foo

\item {} 
Item Bar

\end{itemize}

Or automatically numbered:
\begin{enumerate}
\item {} 
Item 1

\item {} 
Item 2

\end{enumerate}


\subsection{Inline Markup}
\label{resumen:inline-markup}
Words can have \emph{emphasis in italics} or be \textbf{bold} and you can define
code samples with back quotes, like when you talk about a command: \code{sudo}
gives you super user powers!



\renewcommand{\indexname}{Índice}
\printindex
\end{document}
